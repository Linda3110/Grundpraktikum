\documentclass[a4paper]{scrartcl}
\usepackage[utf8]{inputenc}

\begin{document}
\section*{Berechnung von Messunsicherheiten}

\subsection*{Aufgabe 1:}
\textbf{a)} Der Mittelwert ist der durchschnittliche Wert von gegeben Messgrößen.
\\ \textbf{b)} Die Standardabweichung gibt die durchschnittliche Entfernung aller gemessenen Werte vom Durchschnitt an. Sie ist ein Maß für die Streubreite um den Mittelwert.
\\ \textbf{c)} Die Streuung der Messwerte und der Fehler des Mittelwertes unterscheiden sich darin, dass die Standartabweichung angibt, wie sehr die einzelnen Stichproben, um ihren Mittelwert streuen und der Fehler des Mittelwertes die mittlere Abweichung des Mittelwertes einer Stichprobe vom tatsächlichen Mittelwert angibt.

\subsection*{Aufgabe 2:}
$\sigma_u=10\frac{m}{s}$\qquad
$\Delta\overline{k_1}=\pm3\frac{m}{s}$\qquad
$\Delta\overline{k_2}=\pm0,5\frac{m}{s}$\\
\\Es gilt:\qquad
$\Delta\overline{k}=\frac{\sigma_u}{\sqrt{n}}$\qquad
$(=)$\qquad
$n=(\frac{\sigma_u}{\Delta\overline{k}})^{2}$\\
\\Also ergibt sich daraus:\qquad
$n_1=(\frac{10\frac{m}{s}}{\pm3\frac{m}{s}})^{2}=11,\overline{1}\approx11$\qquad
$n_2=(\frac{10\frac{m}{s}}{\pm0,5\frac{m}{s}})^{2}=400$
\end{document}

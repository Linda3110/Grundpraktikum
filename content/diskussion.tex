\section{Diskussion}
\label{sec:Diskussion}
\noindent Nun werden die Versuchsergebnisse aus dem Kapitel \ref{sec:Auswertung} diskutiert. Um die Abweichung zwischen Theorie- und experimentell bestimmten Werten zu berechnen wird die folgende Formel verwendet
\begin{center}
Abweichung $ = \frac{|Theo.Wert\,-\,Exp.Wert|}{Theo.Wert}\cdot 100\%$. 
\end{center}

\noindent Im Folgenden werden die experimentell bestimmte Lebensdauer $\tau$ des Myons mit dem Literaturwert verglichen. 

\begin{table}[H]
\centering
\begin{tabular}{lll}
$\tau_{exp.}$   &=& $(2,043\pm0,028)\,\mathrm{µs}$ \\
$\tau_{Lit.}$   &=& $ 2,1969811\,\mathrm{µs} $ [3]\\
\end{tabular}
\end{table}

\begin{center}
Abweichung $=7,01\,\%$
\end{center}

\noindent Die Abweichung zwischen den beiden Werten ist zwar noch relativ gering, aber der Literaturwert liegt trotzdem deutlich außerhalb des Fehlerbereichs. Eine mögliche Ursache könnte die Zeit-Kanal Kalibrierung sein, da die experimentell bestimmten Lebensdauer $\tau$ nicht nur von den Messergebnissen abhängig ist. Zusätzlich existiert natürlich noch die Untergrundrate, die z.B. dadurch entstehen kann, dass unmittelbar nach Eintreffen eines Myons, ein weiteres Myon innerhalb der Suchzeit eintritt und somit das Stopsignal auslöst (siehe Kapitel \ref{sec:Untergrundrate}). Im Folgenden werden die auf Basis der Ausgleichsrechnung bestimmte Untergrundrate mit der theoretisch berechneten Untergrundrate verglichen.

\begin{table}[H]
\centering
\begin{tabular}{lll}
$U_{exp.}$    &=& $(3,7\pm0,8)$ pro Kanal\\
$U_{theo.}$   &=& $1,27$ pro Kanal\\
\end{tabular}
\end{table}

\begin{center}
Abweichung $=191,34\,\%$
\end{center}

\noindent Die theoretisch berechnete Untergrundrate weicht deutlich von der experimentell bestimmten Untergrundrate ab. Dies könnte eventuell die Abweichung der Lebensdauer erklären, da die zu erwartende Untergrundrate viel geringer ist, als die tatsächliche.